\documentclass[a4paper]{article}

\usepackage[portuguese]{babel}
\usepackage[utf8]{inputenc}
\usepackage[T1]{fontenc}

\newcommand{\documentTitle}{Photorealistic Graphics - Photon mapping} %Macro definition
\newcommand{\documentAuthors}{João Rafael (2008111876, jprafael@student.dei.uc.pt) \and José Ribeiro (2008112181, jbaia@student.dei.uc.pt)} %Macro definition

\title{\documentTitle}
\author{\documentAuthors{}}

\usepackage{hyperref}
\hypersetup{
	pdftitle = \documentTitle
	,pdfauthor = \documentAuthors
	,pdfsubject = {Computer Graphics Project \#2 Report}
	,pdfkeywords = {Computer Graphics Project} {Photon Mapping} {Raytracing}
	,pdfborder = {0 0 0}
}

\usepackage{subfig}
\usepackage{amsmath}
\usepackage{wrapfig}
\usepackage{array}
\usepackage{anysize}
\usepackage{lscape}
\usepackage[pdftex]{graphicx}
\usepackage{longtable}
\usepackage{multirow}
\usepackage[table]{xcolor}

\marginsize{3.5cm}{3.5cm}{3cm}{3cm}

\makeatletter

\begin{document}
\renewcommand{\figurename}{Figure}
\maketitle
\cleardoublepage

\tableofcontents
\cleardoublepage

\setlength{\parindent}{1cm}
\setlength{\parskip}{0.3cm}

\section{Introduction}
\indent \indent 

\cleardoublepage
\section{Implementation}
\indent \indent 

\cleardoublepage
\subsection{Primitives}
\indent \indent 

\subsection{Anti-Alliasing}
\indent \indent

\cleardoublepage
\subsection{Reflection}
\subsubsection{Specular}
\indent \indent 

\subsubsection{Diffuse}
\indent \indent No modelo físico a componente difusa de um material corresponde à reflecção da luz numa superfície
que contém micro-ruído. A luz ao insidir nesta superfície é espalhada para várias direcções. 

\indent Este fenómeno é facilmente reproduzido na fase de \emph{photon-mapping}.
No entanto, na fase de \emph{ray-tracing} seria necessário projectar, para cada reflecção, vários ráios, o que
é computacionalmente muito exigente.

\indent Nesta fase, é portanto, utilizada uma diferente abordagem: a cor difusa inferida a partir da radiosidade local,
isto é, pela intensidade\footnote[1]{A importância de cada fotão depende do ângulo formado entre a sua direcção e a da normal à superfície} dos fotões perto do ponto de reflecção.

\indent Um efeito visível com esta técnica é o \emph{color bleeding}: transferência de cor entre duas superfícies
através de reflecções difusas.

\cleardoublepage
\subsection{Refraction}
\subsubsection{Transmission}

\subsubsection{Caustics}

\cleardoublepage
\section{Gallery}
\indent \indent 

\end{document}
